\documentclass[8pt]{article}

\usepackage{fullpage}
\usepackage[margin=.5in]{geometry}
\usepackage{epic}
\usepackage{eepic}
\usepackage{graphicx}
\usepackage{mathtools}
\usepackage{algorithm}
\usepackage[noend]{algpseudocode}
\usepackage{ragged2e}
\usepackage[parfill]{parskip}
\usepackage{amssymb}


\newcommand{\proof}[1]{
{\noindent {\it Proof.} {#1} \rule{2mm}{2mm} \vskip \belowdisplayskip}
}

\DeclarePairedDelimiter\ceil{\lceil}{\rceil}
\DeclarePairedDelimiter\floor{\lfloor}{\rfloor}


\newtheorem{lemma}{Lemma}[section]
\newtheorem{theorem}[lemma]{Theorem}
\newtheorem{claim}[lemma]{Claim}
\newtheorem{definition}[lemma]{Definition}
\newtheorem{corollary}[lemma]{Corollary}

\begin{document}
\hfill \small{\today} \\
\setlength{\fboxrule}{.5mm}\setlength{\fboxsep}{1.2mm}
\newlength{\boxlength}\setlength{\boxlength}{\textwidth}
\addtolength{\boxlength}{-4mm}
\begin{center}\framebox{\parbox{\boxlength}{\bf
\center{CS 577 - Homework 11}
\center{Sejal Chauhan, Vinothkumar Siddharth, Mihir Shete}
}}\end{center}
\vspace{5mm}

\section{Graded written problem}

\textbf{Part a:} Two Travelling Turkey Problem (TTTP) as mentioned in the written problem has the following contraints:
\begin{itemize}
\item Each city is visited at least once
\item Number of distinct routes has to be as small as possible
\item Each city can be visited any number of times
\item Path starts and ends at the same city
\item Maximum of the total effort of the two turkeys is minimized
\end{itemize}

To formally describe the problem we will consider a complete graph $G = (V, E)$ with $2$ edge-weight functions $w_a: E \rightarrow \mathbb{R}$ and $w_h: E \rightarrow \mathbb{R}$. Where $w_a(e)$ represents the effort taken by \textit{Abe} to travel along the edge $e$ from a city on one end of the edge to the other, while $w_h(e)$ represents the effort required by \textit{Honest}

With the above definition for the graph $G$ we can describe the decision version of the problem as:

For the graph G and $k \in \mathbb{R}$, report \textit{yes} if there exists a spanning tree $S$ such that $MAX(\sum_{\forall e \in S}{w_a(e)}, \sum_{\forall e \in S}{w_h(e)})$ is at-most $k$, else report \textit{no}.

\textbf{Part b:} Decision problem is NP-complete

To prove that TTTP decision problem is NP-complete, we will show that Partition Set(PS) that is a known NP-complete problem, can be reduced to TTTP problem.
Given a set $S$ of Real numbers and the sum of all elements in $S$ as $2*w$, we compute a graph $G$, such that $G$ has a spanning tree $ST$ with $MAX(\sum_{\forall e \in ST}{w_a(e)}, \sum_{\forall e \in ST}{w_h(e)}) \leq w$ only if we can Partition the set using PS algorithm.

To transform the elements of $S$ to the graph $G$ we create a root node, and for every element $i$ in $S$ and add two vertices with edge weight as ($w_a = S[i]$, $w_h = 0$) and ($w_a = 0$, $w_h = S[i]$) to the root note. The edge weight between these two vertices is (0,0). Construct this gadget with all the elements in $S$ and make it a complete graph by connecting all the unconnected vertices with edge weights ($w_a = \infty$,$w_h = \infty$).

Now, if the PS-decision problem returns that \textit{yes} we can partition the set $S$ in $2$ subsets such that each subset has sum $w$ then it is trivial to see that the TTTP-decision problem can always find a spanning tree in the graph $G$ such that the elements in the $2$ subsets correspond to effort of each of the turkeys in the spanning tree, and the TTTP-decision problem will also return \textit{yes}

If the PS-decision problem returns \textit{no}, then consider a spanning tree made of edges whose weights are not $\infty$. Such a spanning tree will have all the elements in the set $S$ either as the edge weights for one turkey or the other. In such a case we can see that the sum of edge-weights for the spanning tree for both the turkeys will be at least $2*w$, now since we cannot \textit{Partition} these weights the sum of weights of edges for one turkey will be less than $w$ and so for the other turkey it will be greater than $w$ and the TTTP-decision problem will also return \textit{no} by definition.

Thus, PS-decision can be reduced from TTTP-decision problem and so TTTP-decision problem is NP-complete.



\end{document}
