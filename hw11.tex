\documentclass[8pt]{article}

\usepackage{fullpage}
\usepackage[margin=.5in]{geometry}
\usepackage{epic}
\usepackage{eepic}
\usepackage{graphicx}
\usepackage{mathtools}
\usepackage{algorithm}
\usepackage[noend]{algpseudocode}
\usepackage{ragged2e}
\usepackage[parfill]{parskip}

\newcommand{\proof}[1]{
{\noindent {\it Proof.} {#1} \rule{2mm}{2mm} \vskip \belowdisplayskip}
}

\DeclarePairedDelimiter\ceil{\lceil}{\rceil}
\DeclarePairedDelimiter\floor{\lfloor}{\rfloor}


\newtheorem{lemma}{Lemma}[section]
\newtheorem{theorem}[lemma]{Theorem}
\newtheorem{claim}[lemma]{Claim}
\newtheorem{definition}[lemma]{Definition}
\newtheorem{corollary}[lemma]{Corollary}

\begin{document}
\hfill \small{\today} \\
\setlength{\fboxrule}{.5mm}\setlength{\fboxsep}{1.2mm}
\newlength{\boxlength}\setlength{\boxlength}{\textwidth}
\addtolength{\boxlength}{-4mm}
\begin{center}\framebox{\parbox{\boxlength}{\bf
\center{CS 577 - Homework 11}
\center{Sejal Chauhan, Vinothkumar Siddharth, Mihir Shete}
}}\end{center}
\vspace{5mm}

\section{Graded written problem}

\textbf{Part a:} Two Travelling Turkey Problem (TTTP) as mentioned in the written problem has the following contraints:
\begin{itemize}
\item Each city is visited at least once
\item Number of distinct routes has to be as small as possible
\item Each city can be visited any number of times
\item Path starts and ends at the same city (DC)
\item Maximum of the total effort of the two turkeys is minimized
\end{itemize}

Formally we can describe the desicion version of the problem as :

Given a doubly-edged complete graph G with different edge weights, find a spanning tree which has minimum cost for the maximum of the two costs among all possible spanning trees. For the graph G, find if there exists a spanning tree whose cost is atmost k.
It can be checked in polynomial time, if such a spanning tree exists. Else it will report that no solution exists. Hence the above formulization fits in the framework of NP-complete problems as described in the lecture notes. This is a NP-complete decision problem under polynomial-time reductions. 

\textbf{Part b:} Decision problem is NP - complete

The NP problem of subsetsum(SS) can be reduced to TTTP as below:


\end{document}
