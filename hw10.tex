\documentclass[8pt]{article}

\usepackage{fullpage}
\usepackage[margin=.5in]{geometry}
\usepackage{epic}
\usepackage{eepic}
\usepackage{graphicx}
\usepackage{mathtools}
\usepackage{algorithm}
\usepackage[noend]{algpseudocode}
\usepackage{ragged2e}
\usepackage[parfill]{parskip}

\newcommand{\proof}[1]{
{\noindent {\it Proof.} {#1} \rule{2mm}{2mm} \vskip \belowdisplayskip}
}

\DeclarePairedDelimiter\ceil{\lceil}{\rceil}
\DeclarePairedDelimiter\floor{\lfloor}{\rfloor}


\newtheorem{lemma}{Lemma}[section]
\newtheorem{theorem}[lemma]{Theorem}
\newtheorem{claim}[lemma]{Claim}
\newtheorem{definition}[lemma]{Definition}
\newtheorem{corollary}[lemma]{Corollary}

\begin{document}
\hfill \small{\today} \\
\setlength{\fboxrule}{.5mm}\setlength{\fboxsep}{1.2mm}
\newlength{\boxlength}\setlength{\boxlength}{\textwidth}
\addtolength{\boxlength}{-4mm}
\begin{center}\framebox{\parbox{\boxlength}{\bf
\center{CS 577 - Homework 10}
\center{Sejal Chauhan, Vinothkumar Siddharth, Mihir Shete}
}}\end{center}
\vspace{5mm}

\section{Graded written problem}

\textbf{Part a:} The Travelling Turkey Problem (TTP-3) as mentioned in the written problem has the following constraints:
\begin{itemize}
\item Each city is visited at least once
\item Number of distinct routes has to be the a small as possible
\item Each city is visited no more than 3 times
\item Path starts and ends at the same city (DC)
\end{itemize}

A city is said to be \textit{visited} if the turkey arrives at the city from some other city. So if the turkey
starts from DC, then DC wont be considered as visited till it comes back to DC from some other city. Also we
note that the turkey can visit any city from every other city, so we will assume that we have a complete graph
with weights along each edge to denote the effort to from city on one end of the edge to the other. To satisfy
the constraint that each city can be \textit{visited} no more than 3 times our solution graph should not have a vertex with 
degree more than 3. Now since we have to visit every city once in a complete graph with the least number of routes
and since each route is represented by an edge, we can say that the Minimum Spanning Tree of the graph with the
constraint that no node in the tree has degree more than 3 is the desired solution if the weight of the edges for
such a tree is smallest. More formally, we can describe the problem as:

Given a complete weighted graph, find a spanning tree in which every vertex has degree at most 3 and which has minimum
cost among all possible spanning trees.

Valid solutions to the above problem can be denoted by strings of length proportional to the number of cities. It is
also trivial to see that the validity and objective value of such a solution string can be checked in polynomial time.
Hence the above formulization fits in the framework of NP-optimization problems as descibed in the lecture notes. \\

\textbf{Part b:} polynomial-time reduce the Circuit-TSP-optimization to TTP-3-optimization

Some terms we will use while doing the reductions: \\
TTP-3: Travelling Turkey Problem where each city can be visited no more than 3 times \\
TTP-2: Travelling Turkey Problem where each city can be visited no more than 2 times

We will prove that Circuit-TSP can be reduced to TTP-3 in 2 steps. First we will reduce Path-TSP to TTP-2, and from 
the equivalence in the Practice problem we wil claim that since Path-TSP is equivalent to Circuit-TSP we can reduce
Circuit-TSP to TTP-2. In the next step we will reduce TTP-2 to TTP-3.

\textbf{\textit{Path-TSP-optimization $\leq$ TTP-2-optimization}}

Given an instance of Path-TSP with graph G. Run TTP-2 on the graph G, 
which returns a MST of G of minimal cost such that degree of no vertex in the tree is more than 2.
(Note that MST will always exist in the complete graph) The path in G is represented by the MST and it starts and 
ends in the leaf nodes of the MST. This is true because we will have exactly 2 leaf nodes in the MST, leaf nodes have degee of one and they
will represent the start and end of our path. We cannot have third leaf node as it will imply that atleast
one vertex in MST has a degree of 3. We claim that P is a minimal-cost path in G or no path exists in G that
visits each vertex exactly once.

Note that in the method used to construct P, we start from one of the leaf nodes of the MST and cover
all the nodes till we reach the other leaf. Also note that since MST covers all the vertices we will
be visiting each vertex once when we travel from one leaf to the other. There will be exactly one path
from the leaf to the other beacause apart from the leaf all other vertices in MST will have degree 2
(The cannot have degree 1 because they will be leaves then and we cannot have more than 2 leaves).
The cost of this path will be equivalent to the cost of MST and since our MST has the minimum cost, our
path will also have the minimum possible cost.

Now, in the other direction, note that for any TSP path X in G, starting at some vertex s and
ending at some vertex t, we can construct a TTP-2 MST Y in G by considering any non-terminal vertex
as root.

Finally, since the MST to Path conversion can be done in linear time, the reduction can be computed in polynomial time.


\textbf{\textit{TTP-2-optimization $\leq$ TTP-3-optimization}}

Given that we to run an instance of TTP-2-optimization with graph G, construct a copy of G, G'. If G has $n$ vertices
then add $n$ new vertices to graph G'. Now, connect each new vertex in G' with just one vertex in G' which was obtained
from G with an edge of weight 0. Now, run TTP-3 on G' to obtain an MST $M_3$. Contruct a MST $M_2$ of G by removing all the newly
added vertices in G' and their associated edges from $M_3$. We now claim than $M_2$ is the MST in G with minimum cost.

Note that in the method used to construct $M_2$, we removed edges of cost 0 from $M_3$, so $M_2$ has the
same cost as $M_3$. Further note that since $M_3$ visits every vertex of G, and $M_2$ visits every vertex 
$M_3$ visits other than the newly added vertices in G', which do not exist in G, $M_2$ visits every vertex of G.
Also, note that the newly added vertices have a degree of 1 and so they are the leaves of $M_3$ and so on removing
them and their associated edges we will get a tree only.
Finally, since this construction did not depend on $M_3$ being minimal, this shows for
every TTP-3 circuit in G' we can construct a TTP-2 circuit in G of equivalent cost.
Now, in the other direction, note that for any TTP-2 circuit X in G, we can construct a TTP-3 circuit Y in G' of
equivalent cost by adding $n$ new vertices and connecting them to one vertex in G with an edge of cost 0.
This equivalence shows that if no TTP-3 circuit exists in G', then no TTP-2 cicuit exists in G, otherwise,
the minimal cost TTP-3 circuit in G' is the same cost as the minimal cost TTP-2 circuit in G, completing the reduction.

Finally, since constructing the graph G' and removing edges from the minimal cost MST $M_3$ in G' can be done in linear time,
the reduction can be computed in polynomial time.

\textbf{Part c:} Formulate a decision version of the Turkey Tour Problem, and polynomial-time reduce it to and from the optimization version

TTP-3-decision problem can be formulated as follows:
Given a complete weighted graph, find if a spanning tree exists in which every vertex has degree at most 3 and which has the cost of atmost C.

\textbf{\textit{TTP-3-optimization $\leq$ TTP-3-decision}}

Given the decision oracle we need to return a MST with minimal cost. Given a graph G and cost C the decision oracle will just report if MST
is possible with cost atmost C. We will proceed with the reduction in following steps:
\begin{itemize}
\item start with cost as sum of all edges and do a binary search using the decision oracle to find the minimum cost $C_{min}$
\item now iterate over each edge in the connected graph and remove it to form graph G', repeat the above step for G' to find $C'_{min}$, if
    $C'_{min} \textless C_{min}$ the removed edge has to be included in the MST for the optimization problem.
\item we can stop when we have $n - 1$ edges. 
\end{itemize}
If sum of all edges is $S$ then the above logic is bounded by $m*S*\log{S}$

\textbf{\textit{TTP-3-decision $\leq$ TTP-3-optimization}}

Given the optimization oracle which will return the MST with least cost. Once we run the optimization oracle and have the MST we calculate its cost,
say its is $C_{min}$. If $C$ is the input cost in the decision problem we should return \textit{no } if $C \textless C_{min}$ and \textit{yes }otherwise

\end{document}
