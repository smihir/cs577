\documentclass[8pt]{article}

\usepackage{fullpage}
\usepackage[margin=.7in]{geometry}
\usepackage{epic}
\usepackage{eepic}
\usepackage{graphicx}
\usepackage{mathtools}
\usepackage{algorithm}
\usepackage[noend]{algpseudocode}
\usepackage{ragged2e}
\usepackage[parfill]{parskip}

\newcommand{\proof}[1]{
{\noindent {\it Proof.} {#1} \rule{2mm}{2mm} \vskip \belowdisplayskip}
}

\DeclarePairedDelimiter\ceil{\lceil}{\rceil}
\DeclarePairedDelimiter\floor{\lfloor}{\rfloor}


\newtheorem{lemma}{Lemma}[section]
\newtheorem{theorem}[lemma]{Theorem}
\newtheorem{claim}[lemma]{Claim}
\newtheorem{definition}[lemma]{Definition}
\newtheorem{corollary}[lemma]{Corollary}

%\setlength{\oddsidemargin}{0in}
%\setlength{\topmargin}{0in}
%\setlength{\textwidth}{6.5in}
%\setlength{\textheight}{8.5in}

\begin{document}

\hfill \small{\today} \\
\setlength{\fboxrule}{.5mm}\setlength{\fboxsep}{1.2mm}
\newlength{\boxlength}\setlength{\boxlength}{\textwidth}
\addtolength{\boxlength}{-4mm}
\begin{center}\framebox{\parbox{\boxlength}{\bf
\center{CS 577 - Homework 3}
\center{Sejal Chauhan, Vinothkumar Siddharth, Mihir Shete}
}}\end{center}
\vspace{5mm}

\section{Graded written problem}

\textbf{Input:} A sequence of n real numbers $a_1$, $a_2$,..., $a_n$ and a corresponding sequence of weights $w_1$, $w_2$, . . . , $w_n$. The weights are nonnegative reals that add up to 1, i.e. $\sum_n^{i=1} w_i = 1$. \\
\\
\textbf{Output:} The weighted median of the sequence is the number $a_k$ such that
$\sum_{a_i \textless a_k} w_i \textless 1/2$ and $\sum_{a_i \textless= a_k} w_i \geq 1/2$

\begin{algorithm}
\caption{Algorithm to find weighted median}\label{euclid}
\begin{algorithmic}[1]
\Procedure{Weighted-Median}{$A_w^a$}

\If{$length(A_w^a) = 1$}
    \State \Return{A[0]} 
\EndIf

\State $Median \leftarrow \Call{Selection}{A_w^a, \ceil{length(A_w^a)/2}}$ \Comment{The Selection will happen over A}
    \State $Pivot \leftarrow Median$
    \State $L_{w}^{a},R_{w}^{r} \leftarrow \Call{Partition}{A_w^a, Pivot}$

    \If{$\sum w_L$ \textless $1/2$}
        \State $w_{Pivot} \leftarrow \sum w_L + w_{Pivot}$
        \If{$w_{Pivot} \geq 1/2$}
            \State \Return ${Pivot}$
        \EndIf
        \State $\Call{Weighted-Median}{Pivot|R_w^a}$
    \Else
        \State $\Call{Weighted-Median}{L_w^a}$
    \EndIf

\EndProcedure
\end{algorithmic}
\end{algorithm}

\begin{flushleft}
Our Algorithm uses the Selection() procedure discussed in section 6 of the Lecture 
notes on Divide and Conquer. When \textit{Weighted-Median} is called it will first
find the $\ceil{n/2}$ smallest element in A using the Selection() procedure in linear
time, let's call this value \textit{Median}(because it is actually the median of
sorted elements in A). We will use this value to \textit{Partition} the Array A into
2 arrays L and R such that the Value $a_i^L$ of all elements in L is less than or equal
to value of \textit{Median} and Value $a_i^R$ of all elements in R is greater than \textit{Median}.

If the weight of all elements in L is less than 1/2 and on adding the weight of the \textit{Pivot}
it becomes greater than or equal to 1/2 then \textit{Pivot} is our Weighted Median and
we return the value of the \textit{Pviot}. If the weight of all elements in L is greater
than or equal to 1/2 then we can say that the Weighted Median is present in the array L
and we recursively call \textit{Weigred-Median} on L.

If the aggregate of weight of elements in L and the weight of \textit{Pivot} is less than 1/2
then we know the Weighted Median is in R. So we change the weight of the
\textit{Pivot} as: $w_{Pivot} \leftarrow w_{Pivot} + \sum L_w$. So \textit{Pivot's} weight
now also includes the weight of elements to its Left(i.e all elements less than equal to pivot).

After changing the weight of \textit{Pivot} we will recursively call \textit{Weighted-Median}
procedure on array $\textit{Pivot}|R$. Prepending \textit{Pivot} with added weight to R
assures that we consider the weight of all elements which are lesser than all elements in R
while calculating the Weighted Median from the new sub-array and the Median retured by the
recursive call is actually the Median of original Array A.

\newpage
\subsection{Correctness}
We will proceed to prove the correctness of our algorithm by proving partial correctness
and termination.

\subsubsection{Partial Correctness}
The recursive calls to \textit{Weighted-Median} will be made on subarray of the input.
It is trivial to see that the length of the subarray will never be negative or 0. So
the recursive calls to \textit{Weighted-Median} will always have a valid array as input.

The algorithm returns from 2 places. The first case is trivially true because if the length
of the array is 1 then the only element in the array is the Weighted-Median. In the second
return we see that the sum of all elements to the Left of the \textit{Pivot} in the partitioned
array is less than 1/2 and when we add the weight of the \textit{Pivot} the aggregated weight
becomes $\geq{1/2}$ then we say that the \textit{Pivot} is the Weighted Median which is true
as per definition because all the elements to the left of \textit{Pivot} in the partitioned
array are less than or equal to the \textit{Pivot}.

\subsubsection{Termination}
Let $\mu(length(A))$ be the potential function then we can see that in the recursive calls

\begin{center}
$\mu(length(A_{recursive)}) = \mu(length(A[0...\ceil{n/2}])) OR \mu(length(A_{recursive)}) = \mu(length(A[\ceil{n/2} + 1 ... n - 1]))$
\end{center}

So the potential function will decrease each recursive call and our program is guaranteed to terminate.

\subsection{Complexity Analysis}
Its trivial to see that our recursive calls operate on an input of size $\ceil{n/2} \pm 1$ and in 
worst case our algorithm will go through log(n) recursive calls.

All the steps in our procedure take constant time except \textit{Selection} and \textit{Partition}
which take linear time. So total running time of our alogrithm in worst case will be:
\begin{center}
$c*(n/2^0) + c*(n/2^1) + c*(n/2^2) ... + c*(n/2^{log(n)}) \le c*n*(1 + 1/2 + 1/4 ....)$
\end{center}
The above is a geometric series with sum limiting to 2. Hence, our algo can take c*2*n time in
worst case and so it is linear in n i.e O(n)
\end{flushleft}

\end{document}
