\documentclass[8pt]{article}

\usepackage{fullpage}
\usepackage[margin=.5in]{geometry}
\usepackage{epic}
\usepackage{eepic}
\usepackage{graphicx}
\usepackage{mathtools}
\usepackage{algorithm}
\usepackage[noend]{algpseudocode}
\usepackage{ragged2e}
\usepackage[parfill]{parskip}

\newcommand{\proof}[1]{
{\noindent {\it Proof.} {#1} \rule{2mm}{2mm} \vskip \belowdisplayskip}
}

\DeclarePairedDelimiter\ceil{\lceil}{\rceil}
\DeclarePairedDelimiter\floor{\lfloor}{\rfloor}


\newtheorem{lemma}{Lemma}[section]
\newtheorem{theorem}[lemma]{Theorem}
\newtheorem{claim}[lemma]{Claim}
\newtheorem{definition}[lemma]{Definition}
\newtheorem{corollary}[lemma]{Corollary}

\begin{document}
\hfill \small{\today} \\
\setlength{\fboxrule}{.5mm}\setlength{\fboxsep}{1.2mm}
\newlength{\boxlength}\setlength{\boxlength}{\textwidth}
\addtolength{\boxlength}{-4mm}
\begin{center}\framebox{\parbox{\boxlength}{\bf
\center{CS 577 - Homework 7}
\center{Sejal Chauhan, Vinothkumar Siddharth, Mihir Shete}
}}\end{center}
\vspace{5mm}

\section{Graded written problem}

\textbf{Input:} Given a sheet $A X B$, there are $(a_{i}, b_{i})$, where $i$ is a positive integer, sculptures that can be formed.
\\ \\
\textbf{Output:} Maximize the number of sculptures that can be made from the starting sheet of paper while minimizing frayed edges that must be visible

\subsection{Algorithm}

Basically, we are using a top-down Dynamic programming approach to solve this problem.
The algorithm in pseudocode looks as follows:
\begin{algorithm}
\caption{Maximize the number of sculptures and minimize frayed edges that must be visible.}\label{euclid}
\begin{algorithmic}[1]
\Procedure{Find-Max-Sculptures}{$A$, $B$, $S[1..n]$}

\State $MaxSculpture[A][B] \leftarrow 0$
\State $FrEdges[A][B] \leftarrow 0$

\For{\texttt{i = 1..A}}
    \For{\texttt{j = 1..B}}
        \For{\texttt{s = S[1]..S[n]}}
            \If{$ s_{length} \textless i$ AND $s_{breadth} \textless j$}\Comment{checking if the length and breadth are less than A and B}
                \State $Max1 \leftarrow 1 + MaxSculpture[i-s_{length}][s_{breadth}] + MaxSculpture[i][j-s_{breadth}]$
                \State $FrEdges1 \leftarrow CalFrayedEdges()$
                \State $Max2 \leftarrow 1 + MaxSculpture[i-s_{length}][j] + MaxSculpture[s_{length}][j-s_{breadth}]$
                \State $FrEdges2 \leftarrow CalFrayedEdges()$
            
            \EndIf
            \If{$ s_{length} \textless B$ $1$ AND $s_{breadth} \textless A$ $2$}
                \State $Max3 \leftarrow 1 + MaxSculpture[s_{breadth}][j-s_{length}] + MaxSculpture[i-s_{breadth}][j]$
                \State $FrEdges3 \leftarrow CalFrayedEdges()$
                \State $Max4 \leftarrow 1 + MaxSculpture[i-s_{breadth}][s_{length}] + MaxSculpture[i][j-s_{length}]$
                \State $FrEdges4 \leftarrow CalFrayedEdges()$
            
            \EndIf
        \State $MaxSculpture[i][j] \leftarrow \max {Max1, Max2, Max3, Max4}$
        \State $FrEdge[i][j] \leftarrow \max {FrEdges1, FrEdges2, FrEdges3, FrEdges4}$
        \EndFor
    \EndFor
\EndFor

\State \Return{$\max (MaxSculpture[][])$}

\EndProcedure
\end{algorithmic}
\end{algorithm}

As we can see,  we are using 3 loops in our procedure, first one is used to iterate over the edge with length $A$. The second one will go over all the permutations of length $B$. While considering each permutation we also check if the frayed edges of the total sculptures that are made. So the performance of our algorithm will be $O(n^3).$

\end{document}
